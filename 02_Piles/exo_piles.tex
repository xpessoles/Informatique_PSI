\documentclass[10pt,fleqn]{book} % Default font size and left-justified equations
\usepackage[%
    pdftitle={Informatique : Programmation récursive},
    pdfauthor={Xavier Pessoles}]{hyperref}

\input{../style/new_style}
\input{../style/macros_SII}

\fichetrue
%\fichefalse

\proftrue
%\proffalse

\tdtrue
%\tdfalse

%\courstrue
\coursfalse

\newcommand{\bfsf}[1]{\textbf{\textsl{#1}}}%{\textbf{\textsf{#1}}}

% -------------------------------------
% Déclaration des titres
% -------------------------------------

\def\discipline{Informatique}
\def\xxtete{Informatique}
\def\classe{PSI$\star$}
\def\xxnumpartie{Partie 5}
\def\xxpartie{Algorithmique \& Programmation II}

\def\xxnumchapitre{Chapitre 2}
\def\xxchapitre{\hspace{.12cm} Piles et files}

\def\xxposongletx{2}
\def\xxposonglettext{1.45}
\def\xxposonglety{13}%10

\def\xxonglet{Part. 5 -- Ch. 2}

\def\xxactivite{TD 2}
\def\xxauteur{\textsl{Xavier Pessoles}}

\def\xxcompetences{%
\textsl{%
\textbf{Savoirs et compétences :}
\begin{itemize}[label=\ding{112},font=\color{ocre}] 
\item Alg -- C16 : Piles - Algorithmes de manipulation : fonctions «push» et «pop».
\end{itemize}
}}

\def\xxfigures{}%figues de la page de garde

\def\xxpied{%
Partie 5 -- Algorithmique et Programmation\\
Ch 2 : Piles et files-- \xxactivite%
}

\def\xxtitreexo{Exercices d'applications}
\def\xxsourceexo{\hspace{.2cm} }


\def\espacebandeautitre{0cm}
\setcounter{secnumdepth}{5}
%---------------------------------------------------------------------------
\begin{document}
\input{../style/new_pagegarde}
\vspace{7cm}
\pagestyle{fancy}
\thispagestyle{plain}

\setcounter{secnumdepth}{5}
\def\columnseprulecolor{\color{ocre}}
\setlength{\columnseprule}{0.4pt} 

\ifprof
\else
\begin{multicols}{2}
\fi
%---------------------------------------------------------------------------

Dans tous ces exrcices on utilisera uniquement les primitives \texttt{creer\_pile}, \texttt{est\_vide}, 
\texttt{est\_pleine}, \texttt{empiler}, \texttt{depiler} et \texttt{taille\_pile}.

\section*{Exercice 1}

Écrire une fonction concaténant deux piles. \\
Cette fonction prendra en argument deux piles et placera la première ``sur'' la deuxième. Elle ne renverra rien, mais 
modifiera les deux piles : à la fin, la première pile sera vide et la seconde aura été augmentée de la première. Pour 
cela, on supposera que la seconde pile est assez grande pour recevoir toute la première.


\section*{Corrigé 1}

\begin{python}
def concat (P1,P2) :
    if est_vide(P1) :
	return None
    else :
	x = depiler(P1)
	concat(P1,P2)
	empiler(P2,x)
\end{python} 


\section*{Exercice 2}

Écrire une fonction renvoyant la plus grande somme possible de trois éléments consécutifs d'une pile d'entiers 
naturels. Si la pile a deux éléments ou moins, on renverra 0. On prendra garde de recopier la pile passée en argument 
afin de ne pas la modifier.

\section*{Corrigé 2}

\begin{python}
def somme3 (P) :
    R = P.copy()
    if est_vide(R) :
	return 0
    else :
	x = depiler(R)
	s = somme3(R)
	for i in range(2) :
	if est_vide(R) :
	    return s
	else :
	    x += depiler(R)
	return max(x,s)
	    
\end{python} 


\section*{Exercice 3}

On se donne deux piles $P_1$ et $P_2$, et on se propose de fusionner ces deux piles en une seule pile $P$ de manière 
à respecter les règles suivantes :
\begin{enumerate}
\item Si $P_1=[a_1,a_2,\cdots,a_{n-1},a_n]$ et $P_2=[b_1,b_2,\cdots,b_{p-1},b_p]$, alors 
$P=[\cdots,b_{p-1},a_{n-1},b_p,a_n]$.
\item Si l'une des pilees est vide, alors on n'a plus recours qu'à l'autre, jusqu'à ce que celle-ci aussi soit vide.
\end{enumerate}
Exemples : $P_1=(a_1,a_2,a_3)$ et $P_2=(b_1,b_2,b_3,b_4,b_5)$, alors $P=[b_1,b_2,a_1,b_3,a_2,b_4,a_3,b_5]$.\\

Écrire une fonction python réalisant cela. On prendra garde de recopier les piles passées en argument afin de ne pas la 
modifier.\\
Afin d'être sûr que la nouvelle pile est assez grande pour contenir les deux premières, on pourra commencer par écrire 
une fonction recopiant une pile dans une autre pile de taille donnée.

\section*{Corrigé 3}


\begin{python}
def copie (P,n) :
    """ recopie la pile P dans une pile de taille n.
    préconditon : n > taille_pile(P) """
    l = taille_pile(P)
    R = P.copy()
    assert n>l
    if est_vide(R) :
        return creer_pile(n)
    else :
        x = depiler(R)
        S=copie(R,n)
        empiler(S,x)
        return S

def fusion (P1,P2) :
    n = taille_pile(P1)+taille_pile(P2)
    R1 = P1.copy()
    R2 = P2.copy()
    if est_vide(R1) :
        return copie(R2,n)
    if est_vide(R2) :
        return copie(R1,n)
    else :        
        e1 = depiler(R1)
        e2 = depiler(R2)
        R3 = fusion(R1,R2)
        empiler(R3,e1)
        empiler(R3,e2)
        return R3 
\end{python} 

\end{document}

