\documentclass[10pt,fleqn]{article} % Default font size and left-justified equations
\usepackage[%
    pdftitle={Informatique : Piles et files},
    pdfauthor={Xavier Pessoles}]{hyperref}

\input{../../style/new_style}
\input{../../style/macros_SII}

\fichetrue
%\fichefalse

%\proftrue
\proffalse

\tdtrue
%\tdfalse

%\courstrue
\coursfalse

\newcommand{\bfsf}[1]{\textbf{\textsl{#1}}}

% -------------------------------------
% Déclaration des titres
% -------------------------------------

\def\discipline{Informatique}
\def\xxtete{Informatique}
\def\classe{PSI$\star$}
\def\xxnumpartie{Partie 5}
\def\xxpartie{Algorithmique \& Programmation II}

\def\xxnumchapitre{Chapitre 2}
\def\xxchapitre{\hspace{.12cm} Piles et files}

\def\xxposongletx{2}
\def\xxposonglettext{1.45}
\def\xxposonglety{13}%10

\def\xxonglet{Part. 5 -- Ch. 2}

\def\xxactivite{TD 1}
\def\xxauteur{\textsl{Jean-Pierre Becirspahic}}

\def\xxcompetences{%
\textsl{%
\textbf{Savoirs et compétences :}
\begin{itemize}[label=\ding{112},font=\color{ocre}] 
\item Alg -- C16 : Piles - Algorithmes de manipulation : fonctions «push» et «pop».
\end{itemize}
}}

\def\xxfigures{}%figues de la page de garde

\def\xxpied{%
Partie 5 -- Algorithmique et Programmation\\
Ch 2 : Piles et files-- \xxactivite%
}

\def\xxtitreexo{Exercices d'applications}
\def\xxsourceexo{\hspace{.2cm} Informatique pour tous en CPGE -- \textit{Wack \& al.}}

\def\espacebandeautitre{0cm}
\setcounter{secnumdepth}{5}
%---------------------------------------------------------------------------
\begin{document}
\input{../../style/new_pagegarde}
\vspace{5cm}
\pagestyle{fancy}
\thispagestyle{plain}

\setcounter{secnumdepth}{5}
\def\columnseprulecolor{\color{ocre}}
\setlength{\columnseprule}{0.4pt} 

\ifprof
\else
\begin{multicols}{2}
\fi
%---------------------------------------------------------------------------


\input{P_05_02_Piles_TD_01.tex}


\ifprof
\else
\end{multicols}
\fi
%\begin{thebibliography}{2}
%\bibitem{1}{Patrick Beynet, \textit{Supports de cours de TSI 2}, Lycée Rouvière, Toulon.}
%\bibitem{2}{<< Mandel zool 08 satellite antenna >>. Sous licence CC BY-SA via Wikimedia Commons - 
%\url{https://fr.wikipedia.org/wiki/Ensemble_de_Mandelbrot#/media/File:Mandel_zoom_08_satellite_antenna.jpg}}
%\bibitem{3}{\url{http://www.obside.fr/fractales/pages/Recursif/}}
%\end{thebibliography}
\end{document}

