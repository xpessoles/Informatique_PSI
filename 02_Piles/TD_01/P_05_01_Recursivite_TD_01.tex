Dans l'ensemble des exercices, on considère disposer des fonctions suivantes :
\begin{itemize}
\item \texttt{creer\_pile(t)} : crée une pile de taille \texttt{t};
\item \texttt{est\_vide(p)} : renvoie \texttt{True} si la pile \texttt{p} est vide;
\item \texttt{empiler(p,e)} : empile l'élément \texttt{e} au sommet de la pile \texttt{p};
\item \texttt{depiler(p)} : dépile l'élément au sommet de la pile \texttt{p} et le retourne;
\end{itemize}
\section*{Exercice 1 - La parenthèse inattendue}
Dans cet exercice, on souhaite savoir si une chaîne de caractères est bien parenthésée ou non. 
Une chaîne bien parenthésée est une chaîne vide ou la concaténation de chaînes bien parenthésées. 
\begin{exemple}
Chaînes bien parenthésées :
\begin{itemize}
\item \texttt{"()"},\texttt{"()()"},\texttt{"(())"} et \texttt{"(()())"}.
\end{itemize}
Chaînes mal parenthésées :
\begin{itemize}
\item \texttt{")("},\texttt{"(("},\texttt{"(()"} et \texttt{"())"}.
\end{itemize}
\end{exemple} 

\subparagraph{}\textit{Implémenter la fonction \texttt{parentheses} répondant aux spécifications suivantes : }

\begin{py}
\begin{python}
def parenthese(s):
    "
    Retourne les couples d'indice parenthèse 
    ouvrante, parenthèse fermante.
    Entrée :
     * s(str) : chaîne de caractères bien 
     parenthésée constituée uniquement 
     de parenthèses.
    Sortie : 
     * Affichage des couples d'indices.
    "
\end{python}
\end{py}

\ifprof
\begin{corrige}
~\\
\begin{python}
def parentheses(s):
    p = creer_pile(len(s))
    for i in range(len(s)):
    if s[i] == '(':
        empiler(p, i)
    else:
        if est_vide(p):
            return False
        j = depiler(p)
        print((j, i))
    return est_vide(p)
\end{python}
\end{corrige}
\else
\fi

\subparagraph{}
\textit{Réaliser un programme permettant de savoir si une chaîne de caractères est bien parenthésée. La structure de pile est-elle nécessaire ?}
\ifprof
\begin{corrige}
\end{corrige}
\else
\fi


\subparagraph{}
\textit{Adapter le premier programme pour qu'il puisse traiter des chaînes constituées de parenthèses, de crochets, ou d'accolades. Un mot est alors bien parenthésé si la parenthèse fermante qui correspond à chaque parenthèse ouvrante est du même type.}
\ifprof
\begin{corrige}
\end{corrige}
\else
\fi

\subparagraph{}
\textit{Adapter le programme pour qu’il puisse traiter des mots constitués de parenthèses et
d’autres caractères, qui n’interfèrent pas avec les parenthèses.}
\ifprof
\begin{corrige}
\end{corrige}
\else
\fi


\subparagraph{}
\textit{Écrire une version récursive de la fonction \texttt{parentheses}.}
\ifprof
\begin{corrige}
\end{corrige}
\else
\fi


\section*{Exercice 2 -- Inversion}
\subparagraph*{}
\setcounter{exo}{0}
\textit{Écrire une fonction qui intervertit les deux éléments situés au sommet d’une pile de taille
au moins égale à 2.}
\ifprof
\begin{corrige}
\end{corrige}
\else
\fi


\section*{Exercice 3 -- Dépile le n\ieme}
\setcounter{exo}{0}
\subparagraph*{}
\textit{Écrire une fonction qui dépile et renvoie le troisième élément d’une pile de taille au moins
égale à 3. Les premier et deuxième éléments devront rester au sommet de la pile.}
\ifprof
\begin{corrige}
\end{corrige}
\else
\fi

\section*{Exercice 4 -- Lire  le n\ieme}
\setcounter{exo}{0}
\subparagraph*{}
\textit{Écrire une fonction qui permet de lire (sans l’extraire) le n-ième élément d’une pile. On
prévoira le cas où la pile n’est pas de taille suffisante pour qu’un tel élément existe.}
\ifprof
\begin{corrige}
\end{corrige}
\else
\fi

%\section*{Exercice}
%\setcounter{exo}{0}
%\subparagraph*{}
%\textit{Programmer les fonctions sommet et taille uniquement à l’aide de empiler, depiler et
%est_vide, indépendamment de la réalisation de pile choisie.
%Que peut-on dire de la complexité en temps et en espace de cette fonction taille ?}
%\ifprof
%\begin{corrige}
%\end{corrige}
%\else
%\fi

\section*{Exercice 5 -- Inversion des extrêmes}

\setcounter{exo}{0}
\subparagraph*{}
\textit{Écrire une fonction qui prend une pile non vide en argument et place l’élément situé à
son sommet tout au fond de la pile, en conservant l’ordre des autres éléments.
Quelle est sa complexité en temps et en espace ?}
\ifprof
\begin{corrige}
\end{corrige}
\else
\fi

\section*{Exercice 6 -- Inversion de la pile}
\setcounter{exo}{0}
\subparagraph{}
\textit{Écrire une fonction similaire à \texttt{reversed}, qui prend une pile en argument et renvoie une autre pile constituée des mêmes éléments placés dans l’ordre inverse.}

\subparagraph{}
\textit{ Si l’on s’autorise à détruire la pile fournie, quelle est la complexité en temps et en espace de cette fonction ? Et si on ne s’y autorise pas ?}
\ifprof
\begin{corrige}
\end{corrige}
\else
\fi

\section*{Exercice 7 -- Tu coupes ?}
\setcounter{exo}{0}
\subparagraph*{}
\textit{Écrire une fonction couper qui prend une pile et la coupe en enlevant de son sommet un
certain nombre d’éléments (tirés au hasard) qui sont renvoyés dans une seconde pile.}

\begin{exemple}
Si la pile initiale est \texttt{[1, 2, 3, 4, 5]}, et que le nombre d’éléments retiré vaut 2, alors la pile ne contient plus que \texttt{[1, 2, 3]} et la pile renvoyée contient \texttt{[5,4]}.
\end{exemple}
\ifprof
\begin{corrige}
\end{corrige}
\else
\fi

\section*{Exercice 8 -- Mélange de cartes}
\setcounter{exo}{0}
\subparagraph*{}
\textit{Écrire une fonction \texttt{melange} qui prend en arguments deux piles et qui
mélange leurs éléments dans une troisième pile de la façon suivante : tant qu’une pile au moins n’est pas vide, on retire aléatoirement un élément au sommet d’une des deux piles et on l’empile sur la pile résultat. }
\begin{exemple}
Un mélange possible des piles \texttt{[1, 2, 3]} et \texttt{[5, 4]} est \texttt{[3, 2, 4, 1, 5]}. Note : à l’issue du mélange, les deux piles de départ sont donc vides.
\end{exemple}

\ifprof
\begin{corrige}
\end{corrige}
\else
\fi
%
%\section*{Exercice 9 -- Tour de magie de Gilbreath}
% Construire un paquet de cartes en empilant n fois les mêmes
%$k$ cartes. (Par exemple, pour un paquet de 32 cartes, on empile n = 16 paquets de paires rouge/noir.)
%Couper alors le paquet avec la fonction couper ci-dessus, puis mélanger les deux paquets obtenus à l’aide
%de la fonction melange. On observe alors que le paquet final contient toujours n blocs des mêmes k cartes
%(même si ces dernières peuvent apparaître dans un ordre différent au sein de chaque bloc). Sur l’exemple
%des 16 paquets rouge/noir, on obtient toujours 16 paquets rouge/noir ou noir/rouge.
