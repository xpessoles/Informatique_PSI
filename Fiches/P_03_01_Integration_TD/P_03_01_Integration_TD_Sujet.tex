\documentclass[10pt,fleqn]{article} % Default font size and left-justified equations
\usepackage[%
    pdftitle={Informatique : Tris},
    pdfauthor={Xavier Pessoles}]{hyperref}

\input{../../style/new_style}
\input{../../style/macros_SII}

\fichetrue
%\fichefalse

%\proftrue
\proffalse

\tdtrue
%\tdfalse

%\courstrue
\coursfalse

\newcommand{\bfsf}[1]{\textbf{\textsl{#1}}}

% -------------------------------------
% Déclaration des titres
% -------------------------------------

\def\discipline{Informatique}
\def\xxtete{Informatique}
\def\classe{PSI$\star$}
\def\xxnumpartie{Partie 3}
\def\xxpartie{Ingénierie Numérique \& Simulation}

\def\xxnumchapitre{Chapitre 1}
\def\xxchapitre{\hspace{.12cm} Intégration numérique}

\def\xxposongletx{2}
\def\xxposonglettext{1.45}
\def\xxposonglety{13}%10

\def\xxonglet{Part. 3 -- Ch. 1}

\def\xxactivite{TD 1}
\def\xxauteur{\textsl{Xavier Pessoles} \\ \textsl{Skander Zannad}}%\textsl{Serge Bays} \\ \textsl{Anthony Meurdefroid}}

\def\xxcompetences{%
\textsl{%
\textbf{Savoirs et compétences :}
%\begin{itemize}[label=\ding{112},font=\color{ocre}] 
%\item Alg -- C17 : tris d’un tableau à une dimension de valeurs numériques (tri par insertion, tri rapide, tri fusion).
%\end{itemize}
}}

\def\xxfigures{}%figues de la page de garde

\def\xxpied{%
Partie 3 -- Ingénierie Numérique \& Simulation\\
Ch 1 : Intégration numérique-- \xxactivite%
}

\def\xxtitreexo{Travaux dirigés}
\def\xxsourceexo{Xavier Pessoles \& Cédric Lopez}%\hspace{.2cm} Informatique pour tous en CPGE -- \textit{Wack \& 
% al.}}

\def\espacebandeautitre{0cm}
\setcounter{secnumdepth}{5}
%---------------------------------------------------------------------------
\begin{document}
\input{../../style/new_pagegarde}
\vspace{5cm}
\pagestyle{fancy}
\thispagestyle{plain}

\setcounter{secnumdepth}{5}
\def\columnseprulecolor{\color{ocre}}
\setlength{\columnseprule}{0.4pt} 
%
%\begin{center}
%\begin{tabular}{|p{\linewidth}|}
%\hline
%\begin{center}
%\textbf{Pour l'ensemble de ce TD, il sera nécessaire de télécharger et décompresser le fichier 
%\url{https://goo.gl/cxjnS3} (aussi disponible via le site de la classe).}
%\end{center}\\
%\hline
%\end{tabular}
%\end{center}

\ifprof
\else
\begin{multicols}{2}
\fi
%---------------------------------------------------------------------------





\subsection*{Exercice 1 : Mise hors gel des canalisations d'eau (temps : 45 min – difficulté : $\star\star $)}

La température dans le sol terrestre étant initialement constante, égale à 5\textdegree C , on cherche à déterminer
à quelle profondeur minimale il est nécessaire d’enterrer une canalisation d’eau pour qu’une brusque
chute de la température de sa surface à -15\textdegree C n’entraine pas le gel de cette canalisation après 10 jours.



%\begin{minipage}[c]{.6\linewidth}
Les hypothèses sont les suivantes :
\begin{itemize}
\item la température en un point quelconque du sol et de sa surface à tout instant $t < 0$ est constante et égale à $T_0=278\;K$ ( $\theta_0=5^oC$ ) ;
\item la température à la surface du sol, confondue avec le plan d’équation $z = 0$, passe brutalement à l’instant $t = 0$ , de $T_0 = 278\;K$ à $T_1 =  258\; K$ ($\theta_1 = -15^o C$) et se maintient à cette valeur pendant
$t_f= $10 jours.
\end{itemize}
\begin{center}
\includegraphics[width=.95\linewidth]{images/canalisation}
\end{center}
On peut montrer que la température $T(z, t)$ à la profondeur $z$ et à l’instant $t$ est donnée par la relation suivante :

$$
T(z,t)=T_1 + (T_0-T_1) \text{erf}\left( \dfrac{z}{2\sqrt{Dt}} \right)
$$
où $\text{erf}(x)$ désigne la fonction définie par :
$$
\text{erf}(x) = \dfrac{2}{\sqrt{\pi}}\int^x_0 e^{-u^2} \mathrm{d}u
$$

Données numériques : $D=2,8\cdot 10^{-7} \; m^2\cdot s^{-1}$ (diffusivité thermique du sol terrestre).

\setcounter{subparagraph}{0}

\subparagraph{}
\textit{Écrire une fonction python, appelée \texttt{integrale}, permettant de réaliser d'intégrer une fonction sur un intervalle, en utilisant la méthode du point milieu.}

\subparagraph{}
\textit{Écrire une fonction Python, appelée \texttt{erf}, prenant en paramètre un nombre réel positif ou nul $x$ et retournant la valeur de \texttt{erf(x)}.}

\subparagraph{}
\textit{Écrire une fonction Python, appelée \texttt{Temperature}, prenant en paramètre la profondeur $z$ (exprimée en $m$) et le temps $t$ (exprimé en $s$) et retournant la valeur de la température $T(z, t)$.}

\subparagraph{}
\textit{Écrire un programme Python permettant de créer une liste, nommée \texttt{ListeErreur}, contenant les valeurs de la fonction \texttt{erf(x)} pour $x$ variant par pas de 0,05 dans l’intervalle $[0 ; 2]$.}

\subparagraph{}
\textit{En déduire, à 1 cm près, à quelle profondeur minimale $z_{min}$ il est nécessaire d'enterrer une canalisation d'eau pour qu'une brusque chute de la température de la surface du sol de 5\textdegree C à -15 \textdegree C n'entraine pas le gel de cette canalisation au bout de 10 jours.}



\ifprof
\else
\end{multicols}
\fi
%\begin{thebibliography}{2}
%\bibitem{1}{Patrick Beynet, \textit{Supports de cours de TSI 2}, Lycée Rouvière, Toulon.}
%\bibitem{2}{<< Mandel zool 08 satellite antenna >>. Sous licence CC BY-SA via Wikimedia Commons - 
%\url{https://fr.wikipedia.org/wiki/Ensemble_de_Mandelbrot#/media/File:Mandel_zoom_08_satellite_antenna.jpg}}
%\bibitem{3}{\url{http://www.obside.fr/fractales/pages/Recursif/}}
%\end{thebibliography}
\end{document}

