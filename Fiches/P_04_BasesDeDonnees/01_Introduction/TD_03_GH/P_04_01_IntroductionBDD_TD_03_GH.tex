\documentclass[10pt,fleqn]{article} % Default font size and left-justified equations
\usepackage[%
    pdftitle={Informatique : Bases de données},
    pdfauthor={Xavier Pessoles}]{hyperref}

\input{style/new_style}
\input{style/macros_SII}

\fichetrue
%\fichefalse

\proftrue
\proffalse

%\tdtrue
\tdfalse

%\courstrue
\coursfalse

% -------------------------------------
% Déclaration des titres
% -------------------------------------

\def\discipline{Informatique}
\def\xxtete{Informatique}

\def\classe{Fiche PT}
\def\xxnumpartie{Partie 4}
\def\xxpartie{Base de données}

\def\xxnumchapitre{Chapitre 1}
\def\xxchapitre{\hspace{.12cm} Introduction aux bases de données}

\def\xxposongletx{2}
\def\xxposonglettext{1.45}
\def\xxposonglety{19}%16

\def\xxonglet{Part. 4 -- Ch. 1}

\def\xxactivite{TD 2}
\def\xxauteur{\textsl{Patrick Beynet}}

\def\xxcompetences{%
\textsl{%
\textbf{Savoirs et compétences :}\\
}}

\def\xxfigures{
%incgraphics[width=.8\textwidth]{}%images/prot_01}
}%figues de la page de garde

\def\xxpied{%
Partie 4 -- Bases de données \\
Ch 1 : Introduction aux bases de données -- \xxactivite%
}

%\def\xxtitreexo{Coucou}
%\def\sourceexo{Coucou}f
\setcounter{secnumdepth}{5}
%---------------------------------------------------------------------------


\begin{document}
%\chapterimage{png/Fond_Cin}
\input{style/new_pagegarde}
\vspace{2cm}
\pagestyle{fancy}
\thispagestyle{plain}


\def\columnseprulecolor{\color{ocre}}
\setlength{\columnseprule}{0.4pt} 

\begin{multicols}{2}
\subsection*{Mines Ponts 2016}

\subparagraph{}~\\
Cléfs primaires pour palu : (nom, année); (iso, année).

\subparagraph{}~\\
%\begin{itemize}
%\item requête récupère depuis la table palu toutes les données de l’année 2010:  SELECT * FROM palu WHERE annee = 2010;
%\item qui correspondent à des pays où le nombre de décès dus au paludisme est supérieur ou égal à 1 000 : 
%WHERE deces >= 1000;
%\end{itemize}
\begin{sql}
SELECT * FROM palu 
	WHERE aneee = 2010 AND deces >= 1000;
\end{sql}

\subparagraph{}~\\
%La requête nécessite des 
%Le résultat attendu à le schéma relationnel suivant : (pays, taux d'incidence) ou encore , (pays, cas*100000/pop). Ces informations provenant des deux tables. Il faut donc réaliser une jointure. Cette jointure peut se faire sur le code iso du pays.  

\begin{sql}
SELECT nom,cas*100000/pop
	FROM palu
	JOIN demographie 
	ON palu.iso=demographie.pays
	WHERE annee = 2011 AND periode = 2011;

SELECT nom,cas*100000/pop
	FROM palu
	JOIN demographie 
	ON palu.iso=demographie.pays 
		AND palu.annee=demographie.periode
 	WHERE annee = 2011;
\end{sql}

\subparagraph{}~\\
%Pour avoir la liste des pays classés par ordre décroissant du nombre de cas de paludisme en 2010, la requète est la suivante. (Pas besoin de grouper ici car (nom,annee) est une clef primaire. 
\begin{sql}
SELECT nom, cas
	FROM palu
	WHERE annee = 2010 
	ORDER BY cas DESC
	LIMIT 1 OFFSET 1; 
\end{sql}

\begin{sql}
SELECT nom, MAX(cas)
	FROM palu
	WHERE annee = 2010 AND cas != (
			SELECT MAX(cas)
			FROM palu
			WHERE annee = 2010
			ORDER BY cas DESC)
	ORDER BY cas DESC;
\end{sql}

\begin{sql}
SELECT nom, MAX(cas)
	FROM palu
	WHERE annee = 2010 AND cas != (
			SELECT MAX(cas)
			FROM palu
			WHERE annee = 2010);
\end{sql}


\subparagraph{}~\\

\texttt{sorted(deces2010, key=lambda col: col[1])}


\subsection*{Centrale Supelec 2018}
\setcounter{exo}{0}
\subparagraph{}~\\
%On souhaite des couples (SI\_DIM,nombre de simulations ayant une dimension SI\_DIM). 
\begin{sql}
SELECT SI_DIM, COUNT(*)
	FROM SIMULATION
	GROUP BY SI_DIM;
\end{sql}

\subparagraph{}~\\
%Écrire une requête SQL qui donne, pour chaque simulation, le nombre de rebonds enregistrés et
%la vitesse moyenne des particules qui frappent une paroi.
\begin{sql}
SELECT SI_NUM, COUNT(*), AVG(RE_VIT)
	FROM REBOND
	GROUP BY SI_NUM;
\end{sql}


\subparagraph{}~\\
%Écrire une requête SQL qui, pour une simulation n donnée, calcule, pour chaque paroi, la variation
%de quantité de mouvement due aux chocs des particules sur cette paroi tout au long de
%la simulation. On se rappellera que lors du rebond d’une particule sur une paroi la composante
%de sa vitesse normale à la paroi est inversée, ce qui correspond à une variation de quantité de
%mouvement de 2mjv?j où m désigne la masse de la particule et v? la composante de sa vitesse
%normale à la paroi.
%On cherche des couples RE\_DIR,  (2*PA\_M,RE\_VP).
%Il faut faire une jointure entre les tables REBOND et PARTICULES  sur le PA\_NUM
%pour les . 

\begin{sql}
SELECT  RE_DIR,  SUM(2*PA_M*RE_VP)
	FROM PARTICULE AS PART
	JOIN REBOND AS REB
	ON PART.PA_NUM =  REB.PA_NUM
	WHERE SIMU.SI_NUM = n
	GROUP BY RE_NUM
\end{sql}


\subsection*{X-ENS 2017}
\setcounter{exo}{0}

\subparagraph{}~\\
\subparagraph{}~\\

Exemple pour sélectionne juste les idpoints avec un produit cartésien. 
\begin{sql}
SELECT  M1.idpoint , M2.idpoint 
	FROM MEMBRE as M1, MEMBRE as M2 
	WHERE M1.idensemble="A" and M2.idensemble="B" AND  M1.idpoint = M2.idpoint 
\end{sql}

Pour sélectionner les coordonnées (x,y)
\begin{sql}
SELECT x,y FROM 
	POINTS JOIN 
	(SELECT  M1.idpoint , M2.idpoint 
	FROM MEMBRE as M1, MEMBRE as M2 
	WHERE M1.idensemble="A" and M2.idensemble="B" AND  M1.idpoint = M2.idpoint) AS MM
	ON POINTS.id = MM.idpoint
\end{sql}




\end{multicols}
%\begin{exemple}
%Exemple de ligne : 
%\begin{python}
%'Blinn Lake Seaplane Base,-162.753005981445,55.2515983581543,seaplane_base\n'
%\end{python}
%\end{exemple}

\end{document}


