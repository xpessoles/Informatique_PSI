\documentclass[10pt,fleqn]{book} % Default font size and left-justified equations
\usepackage[%
    pdftitle={Informatique : Programmation récursive},
    pdfauthor={Xavier Pessoles}]{hyperref}

\input{../../style/new_style}
\input{../../style/macros_SII}

\fichetrue
%\fichefalse

\proftrue
%\proffalse

\tdtrue
%\tdfalse

%\courstrue
\coursfalse

\newcommand{\bfsf}[1]{\textbf{\textsl{#1}}}%{\textbf{\textsf{#1}}}

% -------------------------------------
% Déclaration des titres
% -------------------------------------

\def\discipline{Informatique}
\def\xxtete{Informatique}
\def\classe{PT -- PT$\star$}
\def\xxnumpartie{Partie 5}
\def\xxpartie{Algorithmique \& Programmation II}

\def\xxnumchapitre{Chapitre 1}
\def\xxchapitre{\hspace{.12cm} Programmation récursive}

\def\xxposongletx{2}
\def\xxposonglettext{1.45}
\def\xxposonglety{13}%10

\def\xxonglet{Part. 5 -- Ch. 1}

\def\xxactivite{TD 1}
\def\xxauteur{\textsl{Jean-Pierre Becirspahic}}

\def\xxcompetences{%
\textsl{%
\textbf{Savoirs et compétences :}
\begin{itemize}[label=\ding{112},font=\color{ocre}] 
\item Alg -- C15 : Récursivité : avantages et inconvénients.
\end{itemize}
}}

\def\xxfigures{
}%figues de la page de garde

\def\xxpied{%
Partie 5 -- Algorithmique et Programmation II\\
Ch 1 : Programmation récursive -- \xxactivite%
}

\def\xxtitreexo{Exercices d'application}
\def\xxsourceexo{TD d'informatique du Lycée Louis Legrand -- Jean-Pierre Becirspahic\\
\url{http://info-llg.fr/}}
%---------------------------------------------------------------------------
\begin{document}
\input{../../style/new_pagegarde}
\vspace{7cm}
\pagestyle{fancy}
\thispagestyle{plain}

\setcounter{secnumdepth}{5}
\def\columnseprulecolor{\color{ocre}}
\setlength{\columnseprule}{0.4pt} 

\ifprof
\else
\begin{multicols}{2}
\fi
%---------------------------------------------------------------------------

\input{P_05_01_Recursivite_TD_01.tex}

\ifprof
\else
\end{multicols}
\fi

%\begin{thebibliography}{2}
%\bibitem{1}{Patrick Beynet, \textit{Supports de cours de TSI 2}, Lycée Rouvière, Toulon.}
%\bibitem{2}{<< Mandel zool 08 satellite antenna >>. Sous licence CC BY-SA via Wikimedia Commons - \url{https://fr.wikipedia.org/wiki/Ensemble_de_Mandelbrot#/media/File:Mandel_zoom_08_satellite_antenna.jpg}}
%\bibitem{3}{\url{http://www.obside.fr/fractales/pages/Recursif/}}
%\end{thebibliography}
\end{document}

