\documentclass[10pt,fleqn]{book} % Default font size and left-justified equations
\usepackage[%
    pdftitle={Informatique : Programmation récursive},
    pdfauthor={Xavier Pessoles}]{hyperref}

\input{../style/new_style}
\input{../style/macros_SII}

\fichetrue
%\fichefalse

\proftrue
%\proffalse

\tdtrue
%\tdfalse

%\courstrue
\coursfalse

\newcommand{\bfsf}[1]{\textbf{\textsl{#1}}}%{\textbf{\textsf{#1}}}

% -------------------------------------
% Déclaration des titres
% -------------------------------------

\def\discipline{Informatique}
\def\xxtete{Informatique}
\def\classe{PT -- PT$\star$}
\def\xxnumpartie{Partie 5}
\def\xxpartie{Algorithmique \& Programmation II}

\def\xxnumchapitre{Chapitre 1}
\def\xxchapitre{\hspace{.12cm} Programmation récursive}

\def\xxposongletx{2}
\def\xxposonglettext{1.45}
\def\xxposonglety{13}%10

\def\xxonglet{Part. 5 -- Ch. 1}

\def\xxactivite{TD 1}
\def\xxauteur{\textsl{Jean-Pierre Becirspahic}}

\def\xxcompetences{%
\textsl{%
\textbf{Savoirs et compétences :}
\begin{itemize}[label=\ding{112},font=\color{ocre}] 
\item Alg -- C15 : Récursivité : avantages et inconvénients.
\end{itemize}
}}

\def\xxfigures{
}%figues de la page de garde

\def\xxpied{%
Partie 5 -- Algorithmique et Programmation II\\
Ch 1 : Programmation récursive -- \xxactivite%
}

\def\xxtitreexo{Exercices d'application}
\def\xxsourceexo{TD d'informatique du Lycée Louis Legrand -- Jean-Pierre Becirspahic\\
\url{http://info-llg.fr/}}

\def\espacebandeautitre{0cm}
%---------------------------------------------------------------------------
\begin{document}
\input{../style/new_pagegarde}
\vspace{7cm}
\pagestyle{fancy}
\thispagestyle{plain}

\setcounter{secnumdepth}{5}
\def\columnseprulecolor{\color{ocre}}
\setlength{\columnseprule}{0.4pt} 

\ifprof
\else
\begin{multicols}{2}
\fi
%---------------------------------------------------------------------------



\section*{Exercice }
On considère la fonction suivante :\\
\begin{python}
def f(x) :
    if x <= 1 :
	return 1
    elif x % 2 == 0 :
	return 2*f(x/2)
    else :
	return 1 + f(x+1)
\end{python}

\begin{enumerate}
\item Montrer que cette fonction récursive se termine pour toute valeur entière 
de l'argument \texttt{x}.
\item Soient $x\in\mathbb N$ et $g(x)=f(x)+x$. À partir du programme précédent, donner 
un programme python qui calcule $g(x)$.
\item En déduire que pour tout $x\in\mathbb N$, $g(x)$ est une puissance de 2.
\end{enumerate}




\section*{Corrigé }

\begin{enumerate}
\item Si $x$ est pair et $> 1$, alors $f (x) = 2f (x/2)$ et $ x/2 < x$. Si $x$ est impair et
 $ > 1$, alors $f (x) = 1 + 2f ((x + 1)/2)$ et $(x + 1)/2 < x$. Donc dans les deux cas
     récursifs, on appelle $f$ en une ou deux étapes avec un argument strictement
     inférieur à $x$. Ceci prouve la terminaison du calcul de $f (x)$ par récurrence
     sur $x$.
\item Soit $g(x) = f (x) + x$. En remplaçant $f (x)$ par $g(x)- x$ dans la définition de
    $ $f et en simplifiant, on obtient un programe calculant $g$ :\\
    
   
\begin{python}
def g(x) :
    if x <= 1 :
	return 1 + x
    elif x % 2 == 0 :
	return 2*g(x/2)
    else :
	return g(x+1)
\end{python} 

\item Avec la question précédente, il est immédiat par récurrence que $g(x)$ est toujours une puissance de 2.

 \end{enumerate}




%\begin{thebibliography}{2}
%\bibitem{1}{Patrick Beynet, \textit{Supports de cours de TSI 2}, Lycée Rouvière, Toulon.}
%\bibitem{2}{<< Mandel zool 08 satellite antenna >>. Sous licence CC BY-SA via Wikimedia Commons - \url{https://fr.wikipedia.org/wiki/Ensemble_de_Mandelbrot#/media/File:Mandel_zoom_08_satellite_antenna.jpg}}
%\bibitem{3}{\url{http://www.obside.fr/fractales/pages/Recursif/}}
%\end{thebibliography}
\end{document}

