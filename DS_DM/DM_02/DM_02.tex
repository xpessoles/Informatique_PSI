\ifprof
\else



Dans la mesure du possible, on justifiera le code demandé par des
invariants.


\section{Opérations utiles sur les listes}

Une liste \texttt{t} étant donnée:

\begin{enumerate}
\item \texttt{len(t)} désigne la longueur de \texttt{t};
\item pour tout \texttt{i} dans \texttt{range(len(t))}, \texttt{t[i]} désigne
  l'élément d'indice \texttt{i}.
\item On peut construire une liste de \texttt{n} éléments tous physiquement
  égaux à \texttt{x} avec l'expression \texttt{[x] * n}.
\item On peut ajouter un élément \texttt{x} à la fin de \texttt{t} par
  l'instruction \texttt{t.append(x)} (la longueur de \texttt{t} augmente alors
  de $1$).
\item On peut ajouter tous les éléments d'une liste \texttt{u} à la fin de
  la liste \texttt{t} par l'instruction t += u.
\end{enumerate}

On supposera que ces opérations sur les listes ont respectivement une
complexité:
\begin{enumerate}
\item $\Theta(1)$
\item $\Theta(1)$
\item $\Theta(\text{\texttt{n}})$
\item $\Theta(1)$ (ce n'est pas tout à fait vrai mais pour ce TP,
  c'est raisonnable)
\item $\Theta(\text{\texttt{len(u)}})$ (ce n'est pas tout à fait vrai mais
  pour ce devoir, c'est raisonnable)
\end{enumerate}

\section{Programmation}
\begin{enumerate}
\item Écrire une fonction \texttt{random\_list(n, k)} construisant une
  liste de \texttt{n} entiers tirés au hasard dans l'intervalle
  \texttt{range(k)}. On pourra utiliser la fonction \texttt{randrange} du
  module \texttt{random} à cet effet.
\item Écrire une fonction \texttt{counting\_sort(k, t)} prenant en
  argument une liste t d'entiers appartenant à \texttt{range(k)} et
  retournant une copie triée de ce tableau. On utilisera
  impérativement l'algorithme suivant:
  \begin{itemize}
  \item On construit un tableau \texttt{u} de taille \texttt{k}
    initialisé avec des $0$.
  \item On parcourt \texttt{t}. Pour chaque valeur \texttt{x} trouvée, on
    incrémente \texttt{u[x]}. À la fin du parcours, pour tout entier
    \texttt{i} de \texttt{range(k)}, \texttt{u[i]} contient donc le nombre d'occurrences
    de \texttt{i} dans \texttt{t}.
  \item Il est alors facile de construire un tableau \texttt{r} trié
    répondant à la question posée.
  \end{itemize}
\item Écrire suivant le même principe une fonction \texttt{bucket\_sort(f,
    k, t)} retournant une copie de t triée suivant le critère
  \texttt{f}. Plus précisément, \texttt{f} doit être une fonction prenant ses
  valeurs dans \texttt{range(k)} et les éléments de la liste résultat sont
  triés par ordre croissant de leurs images par f.

  De plus, on fera en sorte que le tri soit \emph{stable},
  c'est-à-dire que pour tout couple de valeurs \texttt{x} et \texttt{y} ayant
  même image par \texttt{f}, \texttt{x} et \texttt{y} apparaissent dans le même
  ordre dans \texttt{t} et dans la liste triée.\\

Par exemple, si $f$ est la fonction $n\mapsto n^2 \% 5$, et \texttt{t=[0,1,5,6,4,12,10,9,17,2,6,7]}. Nous pouvons 
calculer $f(0)=0$, $f(1)=1$, $f(2)=4$, $f(4)=1$, $f(5)=0$, $f(6)=1$, $f(7)=4$, $f(9)=1$, $f(10)=0$, $f(12)=4$ et 
$f(17)=4$. Donc \texttt{bucket\_sort(f, 4, t)} doit renvoyer \texttt{[0,5,10,1,6,4,6,12,9,17,2,7]}. Le tableau  
\texttt{[5,0,10,1,6,4,9,6,12,17,2,7]} est aussi trié suivant le critère $f$, mais ne peut être renvoyé par 
\texttt{bucket\_sort(f, 4, t)} car les deux premiers éléments ne sont pas dans le même ordre que dans \texttt{t}, or on 
veut un tri stable.

\item On se donne la fonction suivante :\\

\begin{python}
def radix_sort(k, t):
    """Retourne une copie de t triée par ordre croissant.
    t doit contenir des entiers appartenant à range(k**2)
    """
    def lp(x):
        return x // k
    def rp(x):
        return x % k
    u = bucket_sort(rp, k, t)
    r = bucket_sort(lp, k, u)
    return r
\end{python}

Soit \texttt{t=[0,5,10,1,6,4,6,12,9,17,2,7]} et $k=5$. Calculer \texttt{bucket\_sort(rp, k, t)}, et en déduire 
\texttt{radix\_sort(k, t)}.\\
Quelle hypothèse peut-on formuler concernant la fonction \texttt{radix\_sort} ?

\newcounter{saveenum}  \setcounter{saveenum}{\value{enumi}}
\end{enumerate}

\section{Étude de complexité}

\begin{enumerate}
 \setcounter{enumi}{\value{saveenum}}
\item  Expliquer (brièvement)  et  sous quelle  hypothèse pourquoi  la
  complexité de \texttt{random\_list(n, k)}) est un $\Theta(n)$.
\item Justifier que la complexité de \texttt{counting\_sort(k, t)} est un
  $\Theta(\max(n, k))$, où \texttt{n = len(t)}.
\item Justifier que la complexité de \texttt{bucket\_sort(f, k, t)} est un
  $\Theta(\max(n, k))$, où \texttt{n = len(t)}.
\item Quelle est la complexité de \texttt{radix\_sort}?
\item 
En admettant que la fonction \texttt{bucket\_sort} répond bien à l'énoncé,
justifier que \texttt{radix\_sort} trie le tableau donné en argument. On
pourra commencer par regarder le cas où \texttt{k} vaut 10 pour comprendre
ce qu'il se passe et se poser la question: à quelle(s) condition(s)
sur \texttt{lp(x)}, \texttt{lp(y)}, \texttt{rp(x)}, \texttt{rp(y)} a t-on \texttt{x<=y}?
\end{enumerate}

\fi

\ifprof
\section*{Corrigé }

\question\

\begin{python}
def random_list (n, k) :
    t = [] # crée un tableau vide
    for i in range (n) : # rajoute n fois un élémént à t
        t.append( randrange (k) )   # cet élément est pris au hasard
                                    # entre 0 et k-1
    return (t)
\end{python}

\question\

\begin{python}

def comptage (k, t) :
    """ si t est un tableau à valeurs dans range(k), comptage renvoie
    un tableau u de longueur k tel que pour tout i, u[i] vaut le nombre
    d'occurences de i dans t """
    u = [0]*k
    for i in t : # on parcourt t
        u[i] += 1 # on ajoute 1 à u[i], car on a trouvé une occurence
                 # supplémentaire de i
    return (u)

def counting_sort (k, t) :
    u = comptage (k,t)
    b = [] # b contiendra le tableau trié
    for i in range (k) : # k est la longueur de u
        b += [i] * u[i] # l'élément i est ajouté u[i] fois
    return b

# autre méthode en utilisant comptage_cumulé, je vous laisse finir

def comptage_cumule (k, t) :
    """ si t est un tableau à valeurs dans range(k), comptage_cumule renvoie
    un tableau u de longueur k tel que pour tout i, u[i] vaut le nombre
    d'occurences d'éléments inférieurs ou égaux à i dans t """
    u = comptage (k, t)
    for i in range(1,k) : # invariant : pour j de 0 à i, u[j] vaut le nombre
                # d'occurences d'éléments inférieurs ou égaux à j dans t
        u[i] = u[i] + u[i-1] 
    return (u)

\end{python}

\question\

\begin{python}

def bucket_sort (f,k,t) :
    baquets = [None]*k # on crée un tableau à k éléments bidons
    for i in range(k) :
        baquets[i] = [] # on crée k baquets vides
    for x in t :
        baquets[f(x)] += [x] # on ajoute l'élément x dans le baquet f(x)
                # on remarque que les éléments d'un même baquet sont dans
                # le même ordre que dans t
    b = [] # b sera le tableau trié
    for baquet in baquets :
        b += baquet # on rajoute les éléments des baquets les uns
                    # après les autres
    return b # b contient les éléments de t, triés suivantleur image
        # et deux éléments de même image sont dans le même ordre que
        # dans t

# test

k = 10
n = 16
# créons une fonction aléatoire f de range(k) dans range(k)
T = random_list (k,k) # ce tableau T contient les images de notre
    # future f dans l'ordre : T = [f(0),f(1),...,f(k-1)]
def f (x) :
    return T[x] # il suffit d'envoyer x sur T[x]

t = random_list (n,k)
print(T)
print(t)
print(bucket_sort (f,k,t))

\end{python}


\question\
Nous avons rp$(0)=0$, rp$(1)=1$, rp$(2)=2$, rp$(4)=4$, rp$(5)=0$, rp$(6)=1$, rp$(7)=2$, rp$(9)=4$, rp$(10)=0$, 
rp$(12)=2$ et rp$(17)=2$. Donc \texttt{bucket\_sort(rp, 5, t)} doit renvoyer \texttt{[0,5,10,1,6,6,12,17,2,7,4,9]}.\\
Puis, lp$(0)=0$, lp$(1)=0$, lp$(2)=0$, lp$(4)=0$, lp$(5)=1$, lp$(6)=1$, lp$(7)=1$, lp$(9)=1$, lp$(10)=2$, lp$(12)=2$ et 
lp$(17)=3$. Donc \texttt{radix\_sort(5, t)} renvoie \texttt{[0,1,2,4,5,6,6,7,9,10,12,17]}.\\
On peut conjecturer que \texttt{radix\_sort} est un algorithme qui trie une liste d'entiers par ordre croissant.



\question\ Le programme 
\verb|random_list(n, k)| ne contient qu'une boucle, de $n$ 
tours. Or à chaque tour une seule instruction est éxécutée : l'ajout d'un 
élément à la fin d'un tableau.\\
Sous l'hypothèse que l'ajout d'un élément à la fin d'un tableau se fait à 
coût constant, alors la complexité de \verb|random_list(n, k)| est bien un 
$\Theta(n)$.\\

\question\ \verb|counting_sort| utilise ici le programme \verb|comptage|. Il faut 
donc d'abord évaluer la complexité de ce dernier programme.\\
La première instruction de \verb|comptage| est la création d'un tableau de 
longueur $k$ : cette opération a une complexité en $\Theta(k)$. Ensuite, la 
boucle du programme a $n$ tours, et dans chaque tour de boucle on suppose que 
l'opération effectué a un coût constant. Ainsi, le temps d'éxécution de 
\verb|comptage| est encadré par deux fonctions de la forme $an+bk+c$, où 
$a,b,c$ sont des constante. C'est donc la plus grande des valeurs de $n$ et $k$ 
qui est prépondérante : la complexité de \verb|comptage| est donc un 
$\Theta(\max(n,k))$.\\
Les instructions de \verb|counting_sort| sont les suivantes :
\begin{itemize}
\item un appel à  \verb|comptage (k,t)| en $\Theta(\max(n,k))$ ;
\item une boucle de $k$ tours dans laquelle la totalité des
opérations (en prenant en compte tous les tours de boucle en une fois) consiste 
à recopier les éléments des éléments de \verb|u| dans un tableau : or il y 
a $n$ éléments à recopier, donc une complexité $\Theta(n)$.
\end{itemize}
Globalement, la complexité est donc bien $\Theta(\max(n,k))$.

\question\ Les instructions de \verb|bucket_sort| sont les suivantes :
\begin{itemize}
\item création d'un tableau de $k$ éléments, soit une complexité $\Theta(k)$ ;
\item une boucle de $n$ tours contenant chacun une opération unitaire, soit 
une complexité $\Theta(n)$ ;
\item une troisième boucle de $k$ tours dans laquelle à chaque tour on accède à 
un élément d'un tableau : ces accès ont une complexité $\Theta(k)$.\\
De plus,  la totalité des autres
opérations (en prenant en compte tous les tours de boucle en une fois) consiste 
à recopier les éléments des éléments de \verb|baquets| dans un tableau : or il y 
a $n$ éléments à recopier, donc une complexité $\Theta(n)$.
\end{itemize}
Globalement, la complexité est donc bien $\Theta(\max(n,k))$.\\

\question\ Notons \texttt{n} la longueur de \texttt{t}.\\
Remarquons d'abord que si \texttt{p} est un entier compris dans 
\texttt{range(k**2)}, alors \texttt{rp(p)} 
et\texttt{lp(p)} sont dans \texttt{range(k)}.\\
\verb|radix_sort| effectue un premier appel à \verb|bucket_sort|, sur une liste de longueur \texttt{n}, et en utilisant 
une fonction à valeurs dans \texttt{range(k)}. Ainsi, cet appel à une complexité en $\Theta(\max(n,k))$. Il renvoie 
\texttt{u}, qui est encore de longueur \texttt{n}.\\
Ensuite, un second appel à \verb|bucket_sort| est effectué, toujours sur une liste de longueur \texttt{n}, et en 
utilisant une fonction à valeurs dans \texttt{range(k)}. La complexité est encore $\Theta(\max(n,k))$.\\
Globalement, la complexité de \verb|radix_sort| est $\Theta(\max(n,k))$.\\
Si $n$ est ``grand'' et $k$ constant (i.e. si la taille du tableau est plus 
grande que les entiers qu'il contient), ce tri est de complexité linéaire, dans 
tous les cas. Ceci est à comparer à la complexité des tris pour lesquels aucune 
hypothèse n'est faite sur les éléments du tableau à trier, et qui est en 
moyenne $\Theta(n\log n)$ pour les tris les plus performants.\\

\question\ Une méthode naturelle pour trier des entiers et d'utiliser l'ordre 
lexicographique, c'est-à-dire que l'on trie ces entiers en regardant d'abord 
leurs chiffres de poids le plus fort, et en finissant par les unités.\\
On pourrait ainsi mettre dans un même baquet tous les entiers commençant par 9 
(ou $k-1$ si l'on compte en base $k$), puis dans un autre tous ceux commençant 
par 8 ... etc ... jusqu'à ceux commençant par 0 (on suppose que tous les 
entiers manipulés sont écrits avec le même nombre de chiffres).\\
Puis on reprend chaque baquet, et dans chaque baquet on refait la même 
opréation, cette fois en regardant le deuxième chiffre. Puis on réitère en 
regardant le troisième chiffre et ainsi de suite. On se rend compte que le 
nombre de baquets utilisés pour ce tri est considérable (combien précisément 
?).\\
Le tri \verb|radix_sort| trie lui les entiers en commençant par comparer les 
chiffres des unités : on met dans un même baquet tous les éléments finissant 
par 0, puis dans un autre finissant par 1 et ainsi de suite. Si l'on compte en 
base $k$, $k$ baquets sont ainsi utilisés. Puis on remet tous les éléments 
ensemble, en respectant l'ordre des baquets. Et on recommence en regardant le 
deuxième chiffre, etc.\\

Lorsque l'on compte en base $k$ et que les éléments du tableau sont dans 
\verb|range(k**2)| comme dans ce TP, alors chaque élément s'écrit sur deux 
bits. Après la première boucle for de \verb|radix_sort|, les éléments sont 
triés suivant leur chiffre des unités. Après la seconde boucle for, ces entiers 
sont triés suivant le bit de poids fort. Mais puisque le tri \verb|bucket_sort| 
est stable, les entiers ayant le même bit de poids fort sont encore triés 
suivant leur chiffre des unités : le tableau est donc bien trié après 
l'éxécution de ces deux boucles.

\else
\fi
