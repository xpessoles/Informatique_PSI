\documentclass[10pt,fleqn]{article} % Default font size and left-justified equations
\usepackage[%
    pdftitle={Informatique : Principe de la représentation des variables en mémoire},
    pdfauthor={Xavier Pessoles}]{hyperref}

\input{style/new_style}
\input{style/macros_SII}

\fichetrue
%\fichefalse

%\proftrue
\proffalse

%\tdtrue
\tdfalse

%\courstrue
\coursfalse

% -------------------------------------
% Déclaration des titres
% -------------------------------------

\def\discipline{Informatique}
\def\xxtete{Informatique}

\def\classe{PSI $\star$}
\def\xxnumpartie{}%Partie 1}
\def\xxpartie{Projets d'informatique}%Architecture matérielle \& logicielle}

\def\xxnumchapitre{}%Chapitres 3 et 4}
\def\xxchapitre{}%\hspace{.12cm} Représentation des variables dans la mémoire}

\def\xxposongletx{2}
\def\xxposonglettext{1.45}
\def\xxposonglety{19}%16

\def\xxonglet{Projets PSI}%P. 1 -- Ch. 3 \& 4}

\def\xxactivite{Fiche}
\def\xxauteur{\textsl{Xavier Pessoles \&  Skander Zannad}  }

\def\xxcompetences{%
\textsl{%
%\textbf{Savoirs et compétences :}\\
%\noindent \textbf{Résoudre :} à partir des modèles retenus :
%\begin{itemize}[label=\ding{112},font=\color{ocre}] 
%\item choisir une méthode de résolution analytique, graphique, numérique;
%\item mettre en \oe{}uvre une méthode de résolution.
%\end{itemize}
%\begin{itemize}[label=\ding{112},font=\color{ocre}] 
%\item \textit{Rés -- C1.1 :} Loi entrée sortie géométrique et cinématique -- Fermeture géométrique.
%\end{itemize}
%
%\noindent \textit{Mod2 -- C4.1 :} Représentation par schéma bloc.
}}

\def\xxfigures{
%incgraphics[width=.8\textwidth]{}%images/prot_01}
}%figues de la page de garde

\def\xxpied{%
Projets d'informatique \\
%Ch. 3 et 4 : Représentation des variables dans la mémoire -- \xxactivite%
}

\setcounter{secnumdepth}{5}
%---------------------------------------------------------------------------


\begin{document}
%\chapterimage{png/Fond_Cin}
\input{style/new_pagegarde}
\vspace{2cm}
\pagestyle{fancy}
\thispagestyle{plain}


\textbf{Qu'est-ce qu'un projet d'informatique ?}

L'objectif de ce projet est de fournir à vos enseignants un programme en python, sur la thématique de votre choix. Il devra être réalisé en 4 semaines.

\vspace{.5cm}
\textbf{Sur quoi le projet doit-il porter ?}

Le projet doit aborder au moins deux thèmes listés ci-dessous : 
\begin{itemize}
\item du calcul numérique (résolution d'équations différentielles, résolution d'équations stationnaires...);
\item base de données;
\item interface graphique (avec matplotlib ou toute autre bibliothèque d'affichage).
\end{itemize}

\vspace{.5cm}
\textbf{Comment s'organiser ?}

Le projet doit être réalisé par groupe de 2 élèves, vous disposez des séances de TD, de cours et de votre temps libre. Le temps estimé à passer sur le projet est compris entre 10 et 15 heures.

\begin{itemize}
\item Vous devez venir à tous les créneaux d'informatique du jeudi après-midi qui correspondent à votre groupe. 
\item Vous pouvez venir même aux créneaux qui ne correspondent pas à votre groupe.
\item Vous pouvez utiliser votre ordinateur personnel. 
\item Si la salle de TD est trop remplie, vous pouvez utiliser la B108.
\end{itemize}

\vspace{.5cm}
\textbf{Que rendre ?}
\begin{itemize}
\item Vous devez rendre un script python (et éventuellement des fichiers dépendants).
\item Le fichier devra être directement exécutable. 
\item Quelques informations peuvent être éventuellement demandées à l'utilisateur via le shell ou via une boite de dialogue.
\item Le résultat de votre travail doit être visuel.
\end{itemize}

\vspace{.5cm}
\textbf{Quels sont les critères d'évaluation ?}

\begin{itemize}
\item Fournir un script fonctionnel est essentiel.
\item Expérience de vos enseignants :).
\end{itemize}

\vspace{.5cm}
\textbf{Vous n'avez pas d'idées de projets ? Nous en avons quelques unes pour vous :)}
\begin{itemize}
\item Coder un jeu ! (Jeu de dames, bataille navale, etc...). Pourquoi ne pas essayer d'utiliser la bibliothèque pygame ?
\item Coder un algorithme de compression de données : 
\begin{itemize}
\item compression sans perte avec l'algorithme de huffman;
\item compresson d'une image en JPG.
\end{itemize}
\item Coder le dépalcement d'une balle ... ou de plusieurs balles :
\begin{itemize}
\item représentation d'une balle qui rebondit sur le sol jusqu'à rouler,
\item représentation d'une balle en chute libre dans l'air... et qui attérit dans l'eau,
\item représentation de plusieurs balles qui s'entrechoquent dans le plan.
\end{itemize}
\item When zombies attack !
\item Modélisation des lois de déplacement d'un stylet de table traçante en 2D.
\item Modélisation d'évacuation d'une salle.
\item Codage d'information en utilisant le codage de hamming (7,4).
\item Modélisation de la déformation d'un treillis (grue, pont, tour Eiffel...).
\item Traitement d'image : augmentation du contraste, floutage, changement de résolution,
recherche de contours.
\end{itemize}

\vspace{.5cm}
\textbf{Livrables}

\textbf{Vous devez rendre vos fichiers python au plus tard le jeudi 8 février 2017 par mail aux deux adresses suivantes :
\url{xpessoles@lamartin.fr} et \url{szannad@lamartin.fr}.}





\newpage
\def\columnseprulecolor{\color{ocre}}
\setlength{\columnseprule}{0.4pt} 

\end{document}

