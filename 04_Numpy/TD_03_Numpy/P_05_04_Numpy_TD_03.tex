\documentclass[10pt,fleqn]{article} % Default font size and left-justified equations
\usepackage[%
    pdftitle={Informatique : Numpy},
    pdfauthor={Xavier Pessoles}]{hyperref}
    
\input{style/new_style}
\input{style/macros_SII}

\usepackage{multicol}
\fichetrue
%\fichefalse

\proftrue
\proffalse

\tdtrue
%\tdfalse

%\courstrue
\coursfalse

% -------------------------------------
% Déclaration des titres
% -------------------------------------

\def\discipline{Informatique}
\def\xxtete{Informatique}
\def\classe{PSI$\star$}
\def\xxnumpartie{Partie 5}
\def\xxpartie{Algorithmique \& Programmation (Suite)}

\def\xxnumchapitre{Chapitre 4}
\def\xxchapitre{\hspace{.12cm} Utilisation de Numpy}

\def\xxtitreexo{Traitement d'images avec numpy}
\def\xxsourceexo{}%\hspace{.2cm} Informatique pour tous en CPGE -- \textit{Wack \& al.}}

\def\xxposongletx{2}
\def\xxposonglettext{1.45}
\def\xxposonglety{13}%10

\def\xxonglet{Part. 5 -- Ch. 4}

\def\xxactivite{TD -- 03}
\def\xxauteur{\textsl{Xavier Pessoles}}

\def\xxcompetences{%
\textsl{%
\textbf{Savoirs et compétences :}
%\begin{itemize}[label=\ding{112},font=\color{ocre}] 
%\item Alg -- C17 : tris d’un tableau à une dimension de valeurs numériques (tri par insertion, tri %rapide, tri fusion).
%\end{itemize}
}}

\def\xxfigures{}%figues de la page de garde

\def\xxpied{%
Partie 5 -- Algorithmique et Programmation\\
Ch 4 : Utilisation de numpy -- \xxactivite%
}



\setcounter{secnumdepth}{5}
%---------------------------------------------------------------------------


\begin{document}
%\chapterimage{png/Fond_Cin}
\input{style/new_pagegarde}
\vspace{4cm}
\pagestyle{fancy}
\thispagestyle{plain}


\def\columnseprulecolor{\color{ocre}}
\setlength{\columnseprule}{0.4pt} 
\ifprof
\else
\begin{multicols}{2}
\fi

\subsection*{Représentation informatique d'une couleur}
\ifprof
\else
La représentation d'une couleur est déterminée par un choix de codage de l'information
«~couleur~».

On peut choisir d'utiliser 1~bit pour stocker cette
information. On distinguera alors uniquement deux couleurs. Par
exemple du noir (0) et du blanc (1) ou alors du blanc (0) et du noir
(1) ou pourquoi pas du rouge et du noir. 

On peut choisir d'utiliser 1~octet pour stocker cette information. On
distinguera alors $2^8=256$ couleurs différentes. Ces $256$ valeurs
différentes peuvent représenter des niveaux de gris différents (du
blanc au noir ou alors du noir au blanc) ou pourquoi pas du blanc au
rouge en passant par 254 nuances de rose.

Le plus souvent, on utilise 3~octets pour stocker une couleur. On distingue
alors $256^3 = 16\,777\,216$ couleurs différentes. La norme RGB consiste à
séparer ces trois octets et les penser comme
 un vecteur à 3~coordonnées dans l'espace
$R\times G \times B$, où chaque coordonnée peut prendre $2^8=256$
valeurs. 

%\renewcommand{\figurename}{}
%\renewcommand{\thefigure}{}
%\begin{figure}[h]
\begin{center}
  %\centering
  \includegraphics[width=.2\textwidth]{images/theme_image_1_fig_1}
  \includegraphics[width=.2\textwidth]{images/theme_image_1_fig_2}
%  \caption{\ccbysa\ }
  % \caption{By SharkD (Own work), \ccbysa\ }
  % https://commons.wikimedia.org/wiki/File:RGB_color_solid_cube.png
\end{center}

%\begin{center}
  %\centering
%  \includegraphics[width=.4\textwidth]{images/theme_image_1_fig_1}
%  \includegraphics[width=.45\textwidth]{images/theme_image_1_fig_2}
%  \caption{\ccbysa\ }
  % \caption{By SharkD (Own work), \ccbysa\ }
  % https://commons.wikimedia.org/wiki/File:RGB_color_solid_cube.png
%\end{center}


On peut aussi utiliser 4~octets (trois octets pour la couleur et un
pour la transparence).

On peut aussi utiliser des résolutions plus importantes, des
compressions sans pertes. 

Bref, il importe de se référer aux spécifications des fichiers que l'on
utilise~!

\subsection*{Représentation informatique d'une image}

% Nous ne nous intéressons pas ici à la représentation vectorielle des
% images, qui consiste à décrire la construction mathématique d'une
% image. Nous ne nous intéressons pas non plus aux formats compressés
% des images. 
Il existe de nombreuses façons de représenter informatiquement une
image. Certaines sont vectorielles (SVG, PS), compressées (JPG) d'autres
matricielles (PNG). Nous ne nous intéressons ici qu'aux images au
format PNG. Il s'agit d'un ensemble
discret de points ayant chacun une couleur donnée. Ces points
s'appellent des pixels. La résolution de
l'image dépend du nombre de points (nombre de lignes et de colonnes),
et aussi du nombre de bits servant à coder les couleurs.

Une image est stockée dans un tableau numpy, dont le
\texttt{shape} est \texttt{(n,p,3)}, avec $n$ le nombre de lignes, $p$
le nombre de colonnes et $3$ le nombre d'octets pour coder les
couleurs. 

% Le \texttt{dtype} des images que nous utilisons ici est
% \texttt{float32}. 

En plus des bibliothèques habituelles
\begin{python}
import numpy as np
import matplotlib.pyplot as plt
\end{python}
on charge la bibliothèque~:
\begin{python}
import matplotlib.image as mpimg
\end{python}
qui nous permet d'utiliser les instructions~: 
\begin{itemize}
\item 
  \verb#img = mpimg.imread(filename)#~: le fichier PNG \verb#filename#
  est converti en tableau numpy et nommé \verb#img#.
\item 
  \verb#plt.imshow(img)#~: comme \verb#plt.plot(...)#, l'affichage de
  \verb#img# est préparé, et sera utilisé par \verb#plt.show()# ou
  \verb#plt.savefig(filename2)#.
\item
  \verb#mpimg.imsave(filename2, img)# permet d'enregistrer l'image
  (qui n'est pas un schéma python, n'a pas d'axes, de titre etc.)
\end{itemize}

\begin{remark}
  Avec \texttt{matplotlib.image}, les tableaux numpy sont de
  \texttt{dtype} \texttt{float32}, mais ils sont constitués uniquement
  de 256 valeurs flottantes uniformément réparties entre $0$ et $1$.
\end{remark}


\subsection*{Quelques fonctions élémentaires}
Télécharger l'image ***.
\subparagraph{}\textit{Ouvrir l'image avec \\ \texttt{matplotlib.image.imread} et l'afficher.}


\subparagraph{}\textit{Déterminer pour l'image téléchargée la taille, le
  \texttt{dtype}, la couleur du pixel en position (300,300) et les
  valeurs max. et min. des éléments.}

\subparagraph{}\textit{Écrire une deux fonctions \texttt{conv\_1\_2\_255} et \texttt{conv\_255\_2\_1} permettant respectivement :
\begin{itemize}
\item << d'étaler >> chacun des niveaux de couleurs de 0 à 255 (en valeurs entières);
\item << de normer >> chacun des niveaux de couleurs de 0 à 1 (en valeurs flottantes);
\end{itemize}.}

L'histogramme de couleurs représente le nombre de pixels pour chaque couche de couleur. En retouche d'image, la modification de l'histogramme peut par exemple modifier le contraste d'une photo. 

\subparagraph{}\textit{Tracer l'histogramme de chacune des couleurs de l'image.}

\begin{rem}
Il semble être plus rapide de créer la liste du nombre d'occurrences de chaque couleur de pixel puis de tracer cette liste, plutôt que d'utiliser la méthode \texttt{plt.hist}.
\end{rem}






\subsection*{Jouons avec les couleurs, le noir et le blanc}
%\subparagraph{}\textit{En utilisant le codage donné précédemment, << où >> est codée la couleur bleue ?}

\subparagraph{}\textit{Afficher uniquement les trames rouge, verte et bleue (pour la trame de rouge, seuls les pixels rouges seront conservés, les autres seront mis à 0).}

Pour saturer l'image en une couleur, le niveau de cette couleur doit être à 1 dans chacun des pixels de l'image. 

\subparagraph{}\textit{Afficher les images saturées en rouge, vert et bleu.}
\ifprof
\begin{corrige}
\begin{python}
img[:,:,2]=1
\end{python}
\end{corrige}
\else
\fi

\subparagraph{}\textit{Réaliser une inversion des couleurs en remplaçant le niveau $n$ de chaque couleur par $1-n$.}
\ifprof
\begin{corrige}
\begin{python}
img[:,:,:]=1-img[:,:,:]
img=1-img # Autre solution
\end{python}
\end{corrige}
\else
\fi

\vspace{.5cm}

Plusieurs méthodes permettent de transformer une image en couleur en image en noir et blanc. Il est possible d'attribuer à chaque niveau de couleur d'un pixel la moyenne des niveaux de couleur du pixel considéré. 


\subparagraph{}\textit{Transformer l'image en noir et blanc en utilisant la définition précédente. Pour cela, réaliser une fonction \texttt{convert\_nb\_moy} qui permet de pondérer les niveaux de couleurs.}


\subparagraph{}\textit{Réaliser une transformation en noir et blanc en réalisant la moyenne du minimum et du maximum des 3 niveaux.}

\subsection*{Amélioration du contraste}
On va ici modifier le contraste d'une image en niveau de gris. 
Pour cela, on va accentuer les différences entre les niveaux en utilisant une fonction mathématique. 

\subparagraph{}\textit{Donner la représentation des fonctions $f:x \mapsto \sqrt{x}$ et $g:x\mapsto \dfrac{1}{2}\left(1+\sin\left(\pi\left(x-\dfrac{1}{2}\right)\right)\right)$ sur l'intervalle $[0,1]$.}

\subparagraph{}\textit{Modifier le contraste d'une image en noir et blanc en remplaçant les niveaux de gris $x$ par $f(x)$ et $g(x)$. Conclure.}

\subsection*{Convolution}
La méthode de convolution permet, suivant les cas, de flouter ou de détecter des concours. 
Selon cette méthode, le niveau de couleur de chaque pixel est recalculé en fonction de ses voisins. 


On note $M=\begin{pmatrix} a & b & c \\ d & e & f \\ g & h & i \end{pmatrix}$. Chaque pixel $(i,j)$ est alors remplacé ainsi : $\text{img}[x, y] = a \times \text{img}[x-1, y+1] + b \times \text{img}[x, y+1]+ c \times \text{img}[x+1, y+1] + \ldots + i \times \text{img}[x+1, y-1]$.

\subparagraph{}\textit{Écrire la fonction \texttt{convolution(image,matrice)} permettant d'appliquer la convolution à une image.}

\textit{Vous pourrez tester votre fonction avec les matrices suivantes : 
$\begin{pmatrix} 1/8 & 1/8 & 1/8 \\ 1/8 & -1/8 & 1/8 \\ 1/8 & 1/8 & 1/8 \end{pmatrix}$ 
$\begin{pmatrix} 1 & 1 & 1 \\ 1 & -2 & 1 \\ 1 & 1 & 1 \end{pmatrix}$. Quel est l'effet de la convolution sur l'image ?}


\ifprof
\else
\end{multicols}
\fi

\end{document}
\section*{Exercice  -- }


\ifprof 
\else
\fi


\section*{Exercice}
\setcounter{exo}{0}
\subparagraph{}
\textit{}
\ifprof
\begin{corrige}
\end{corrige}
\else
\fi