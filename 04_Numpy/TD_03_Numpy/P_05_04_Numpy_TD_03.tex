\documentclass[10pt,fleqn]{article} % Default font size and left-justified equations
\usepackage[%
    pdftitle={Informatique : Numpy},
    pdfauthor={Xavier Pessoles}]{hyperref}
    
\input{style/new_style}
\input{style/macros_SII}

\usepackage{multicol}
\fichetrue
%\fichefalse

\proftrue
\proffalse

\tdtrue
%\tdfalse

%\courstrue
\coursfalse

% -------------------------------------
% Déclaration des titres
% -------------------------------------

\def\discipline{Informatique}
\def\xxtete{Informatique}
\def\classe{PSI$\star$}
\def\xxnumpartie{Partie 5}
\def\xxpartie{Algorithmique \& Programmation (Suite)}

\def\xxnumchapitre{Chapitre 4}
\def\xxchapitre{\hspace{.12cm} Utilisation de Numpy}

\def\xxtitreexo{Traitement d'images avec numpy}
\def\xxsourceexo{}%\hspace{.2cm} Informatique pour tous en CPGE -- \textit{Wack \& al.}}

\def\xxposongletx{2}
\def\xxposonglettext{1.45}
\def\xxposonglety{13}%10

\def\xxonglet{Part. 5 -- Ch. 4}

\def\xxactivite{TD -- 02}
\def\xxauteur{\textsl{Guillaume Haberer}}

\def\xxcompetences{%
\textsl{%
\textbf{Savoirs et compétences :}
%\begin{itemize}[label=\ding{112},font=\color{ocre}] 
%\item Alg -- C17 : tris d’un tableau à une dimension de valeurs numériques (tri par insertion, tri %rapide, tri fusion).
%\end{itemize}
}}

\def\xxfigures{}%figues de la page de garde

\def\xxpied{%
Partie 5 -- Algorithmique et Programmation\\
Ch 4 : Utilisation de numpy -- \xxactivite%
}



\setcounter{secnumdepth}{5}
%---------------------------------------------------------------------------


\begin{document}
%\chapterimage{png/Fond_Cin}
\input{style/new_pagegarde}
\vspace{5cm}
\pagestyle{fancy}
\thispagestyle{plain}


\def\columnseprulecolor{\color{ocre}}
\setlength{\columnseprule}{0.4pt} 
\ifprof
\else
\begin{multicols}{2}
\fi

\subsection*{Représentation informatique d'une couleur}
\ifprof
\else
La représentation d'une couleur est déterminée par un choix de codage de l'information
«~couleur~».

On peut choisir d'utiliser 1~bit pour stocker cette
information. On distinguera alors uniquement deux couleurs. Par
exemple du noir (0) et du blanc (1) ou alors du blanc (0) et du noir
(1) ou pourquoi pas du rouge et du noir. 

On peut choisir d'utiliser 1~octet pour stocker cette information. On
distinguera alors $2^8=256$ couleurs différentes. Ces $256$ valeurs
différentes peuvent représenter des niveaux de gris différents (du
blanc au noir ou alors du noir au blanc) ou pourquoi pas du blanc au
rouge en passant par 254 nuances de rose.

Le plus souvent, on utilise 3~octets pour stocker une couleur. On distingue
alors $256^3 = 16\,777\,216$ couleurs différentes. La norme RGB consiste à
séparer ces trois octets et les penser comme
 un vecteur à 3~coordonnées dans l'espace
$R\times G \times B$, où chaque coordonnée peut prendre $2^8=256$
valeurs. 

%\renewcommand{\figurename}{}
%\renewcommand{\thefigure}{}
%\begin{figure}[h]
\begin{center}
  %\centering
  \includegraphics[width=.2\textwidth]{images/theme_image_1_fig_1}
  \includegraphics[width=.2\textwidth]{images/theme_image_1_fig_2}
%  \caption{\ccbysa\ }
  % \caption{By SharkD (Own work), \ccbysa\ }
  % https://commons.wikimedia.org/wiki/File:RGB_color_solid_cube.png
\end{center}

%\begin{center}
  %\centering
%  \includegraphics[width=.4\textwidth]{images/theme_image_1_fig_1}
%  \includegraphics[width=.45\textwidth]{images/theme_image_1_fig_2}
%  \caption{\ccbysa\ }
  % \caption{By SharkD (Own work), \ccbysa\ }
  % https://commons.wikimedia.org/wiki/File:RGB_color_solid_cube.png
%\end{center}


On peut aussi utiliser 4~octets (trois octets pour la couleur et un
pour la transparence).

On peut aussi utiliser des résolutions plus importantes, des
compressions sans pertes. 

Bref, il importe de se référer aux spécifications des fichiers que l'on
utilise~!

\subsection*{Représentation informatique d'une image}

% Nous ne nous intéressons pas ici à la représentation vectorielle des
% images, qui consiste à décrire la construction mathématique d'une
% image. Nous ne nous intéressons pas non plus aux formats compressés
% des images. 
Il existe de nombreuses façons de représenter informatiquement une
image. Certaines sont vectorielles (SVG, PS), compressées (JPG) d'autres
matricielles (PNG). Nous ne nous intéressons ici qu'aux images au
format PNG. Il s'agit d'un ensemble
discret de points ayant chacun une couleur donnée. Ces points
s'appellent des pixels. La résolution de
l'image dépend du nombre de points (nombre de lignes et de colonnes),
et aussi du nombre de bits servant à coder les couleurs.

Une image est stockée dans un tableau numpy, dont le
\texttt{shape} est \texttt{(n,p,3)}, avec $n$ le nombre de lignes, $p$
le nombre de colonnes et $3$ le nombre d'octets pour coder les
couleurs. 

% Le \texttt{dtype} des images que nous utilisons ici est
% \texttt{float32}. 

En plus des bibliothèques habituelles
\begin{python}
import numpy as np
import matplotlib.pyplot as plt
\end{python}
on charge la bibliothèque~:
\begin{python}
import matplotlib.image as mpimg
\end{python}
qui nous permet d'utiliser les instructions~: 
\begin{itemize}
\item 
  \verb#img = mpimg.imread(filename)#~: le fichier PNG \verb#filename#
  est converti en tableau numpy et nommé \verb#img#.
\item 
  \verb#plt.imshow(img)#~: comme \verb#plt.plot(...)#, l'affichage de
  \verb#img# est préparé, et sera utilisé par \verb#plt.show()# ou
  \verb#plt.savefig(filename2)#.
\item
  \verb#mpimg.imsave(filename2, img)# permet d'enregistrer l'image
  (qui n'est pas un schéma python, n'a pas d'axes, de titre etc.)
\end{itemize}

\begin{remark}
  Avec \texttt{matplotlib.image}, les tableaux numpy sont de
  \texttt{dtype} \texttt{float32}, mais ils sont constitués uniquement
  de 256 valeurs flottantes uniformément réparties entre $0$ et $1$.
\end{remark}


\subsection*{Quelques fonctions élémentaires}
Télécharger l'image ***.
\subparagraph{}\textit{Ouvrir l'image avec \texttt{matplotlib.image.imread} et l'afficher.}


\subparagraph{}\textit{Déterminer pour l'image \texttt{playmobil.png}~: la taille, le
  \texttt{dtype}, la couleur du  pixel en position (300,300) et les
  valeurs max. et min. des éléments.}




\subsection*{Jouons avec les couleurs}
\subparagraph{}\textit{En utilisant le codage donné précédemment, << où >> est codée la couleur bleue ?}

Pour saturer l'image en bleue, le niveau de bleu doit être à 1 dans chacun des pixels de l'image. 

\subparagraph{}\textit{Réaliser la saturation en bleu.}

\begin{python}
img[:,:,2]=1
\end{python}

\subparagraph{}\textit{Réaliser une inversion des couleurs en remplaçant le niveau $n$ de chaque couleur par $1-n$.}
\begin{python}
img[:,:,:]=1-img[:,:,:]
img=1-img # Autre solution
\end{python}


\ifprof
\begin{corrige}
\begin{python}
#! /usr/local/bin/python3.4
# -*- coding:utf-8 -*-
    
import numpy as np
import matplotlib.pyplot as plt
import matplotlib.image as mpimg

# question (a)
img = mpimg.imread('./theme_image_1_playmobil.png')  

# plt.figure(1)
# plt.imshow(img)
# plt.show()

# question (b)

print(img.shape)
# (564, 1000, 3)
# il y a 564 pixels en hauteur et 1000 en largeur,
# et 3 composantes couleur pour chaque pixel.
 
print(img.dtype)
# float32

print(img[300,300,:])
# [ 0.11764706  0.24705882  0.41568628]

print(img.max())
print(img.min())
# 1.0
# 0.0

# question (c)
# En observant l'image, quitte à zoomer, on voit que le fond n'est pas uniforme.
# Pour répondre précisément à la question, dénombrons les valeurs différentes
# dans la zone en haut à gauche (strictement) du pixel (300,300)
zone = img[0:300,0:300,:]
# couleurs = []
# for i in range(300):
#     for j in range(300):
#         if list(zone[i,j,:]) not in couleurs:
#             couleurs.append(list(zone[i,j,:]))
# print(len(couleurs))
# 3734
# Il y a 3734 couleurs différentes sur les 90 000 pixels considérés.
# Le fond n'est donc clairement pas uniforme,
# même si, à l'oeil, il peut sembler uniforme.

# question (d)

playmobil = mpimg.imread('./theme_image_1_playmobil.png')  
montagne = mpimg.imread('./theme_image_1_montagne.png')

# on choisit un pixel bleu de référence
reference_bleue = playmobil[300,300,:]

# pour décider de la valeur du seuil, correspondant à une distance "faible",
# on fait des essais. 
seuil = .30

# Une première version sans utiliser toutes les fonctionnalités de numpy

# les pixels qui ont une couleur proche du pixel de référence sont remplacés
# par les pixels de montagne correspondants
# les autres pixels (qui correspondent donc aux playmobils) ne sont pas modifiés
n,p,q = img2.shape
for i in range(n):
    for j in range(p):
        if np.sqrt((playmobil[i,j,0]-pixel_ref[0])**2 + (playmobil[i,j,1]-pixel_ref[1])**2
                   + (playmobil[i,j,2]-pixel_ref[2])**2)< seuil :
            playmobil[i,j,:] = img1[i,j,:]

# plt.figure()
# plt.imshow(playmobil)
# plt.show()

# Une deuxième version en utilisant les opérations terme à terme de numpy

# on sélectionne les pixels à une distance inférieure au seuil
# en utilisant un masque
mask = np.sqrt( (playmobil[:,:,0] - reference_bleue[0])**2
               +(playmobil[:,:,1] - reference_bleue[1])**2
               +(playmobil[:,:,2] - reference_bleue[2])**2) < seuil
print(mask.shape)
# (564, 1000)
mask.shape = (564, 1000, 1)
masque_complet = np.concatenate((mask, mask, mask), axis=2)

playmobil_sans_bleu = playmobil * (1 - masque_complet)
# plt.figure(2)
# plt.imshow(playmobil_sans_bleu)
# plt.show()

montagne_decoupee = montagne * masque_complet
# plt.figure(3)
# plt.imshow(montagne_decoupee)
# plt.show()

images_superposees = playmobil_sans_bleu + montagne_decoupee
# plt.figure(4)
# plt.imshow(images_superposees)
# plt.show()

# question (e)
mpimg.imsave("theme_image_1_fake.png", images_superposees)
\end{python}
\end{corrige}
\else
\fi



\section*{Exercice 5 -- Représentation informatique du son.}

\ifprof
\else
Le son est une vibration de l'air, une suite de surpressions et de
dépressions de l'air. L'enregistrement d'un son, c'est
l'enregistrement de cette suite de variations de pression de l'air en
fonction du temps~: on parle de \emph{signal analogique}.

\centerline{\includegraphics[height=3cm]{images/theme_son_2_fig_1}}

Pour représenter informatiquement un tel signal, on discrétise
l'information. On parle d'\emph{échantillonnage} ou de
\emph{numérisation}. Le \emph{signal numérique} est donc la suite des
valeurs du signal pour une succession d'instants bien définis. 

\centerline{\includegraphics[width=\linewidth]{images/theme_son_2_fig_2}}

Lors de la numérisation, il y a une perte d'information plus ou moins
grande selon les valeurs choisies pour~: 
\begin{itemize}
\item 
  la \emph{fréquence d'échantillonnage} ou \emph{sample rate}. C'est le
  nombre de valeurs par seconde, exprimée en Hz. \\
  Le théorème de Shannon stipule que la fréquence d'échantillonnage doit
  être au moins le double de la fréquence maximale contenue dans le
  signal. Les fréquences les plus hautes (aiguës) perçues par l'oreille
  humaine sont de l'ordre de 20~kHz. \\
  On donne quelques valeurs courantes~: DVD~: 48~kHz, CD~: 44100~Hz,
  radio~: 22~kHz, téléphone~: 8~kHz  
  
\item 
  la \emph{résolution} (quantification) ou \emph{bit depth}. C'est le nombre de bits
  servant à coder les valeurs à un instant donné. \\
  Avec 8 bits, seules 256 valeurs différentes sont utilisées, tandis
  qu'avec 16 bits, ce sont 65536 valeurs différentes qui sont
  utilisées, ce qui permet de représenter un son beaucoup plus riche. 
  \\
  On donne quelques valeurs courantes~: DVD~: 24~bits, CD~: 16~bits,
  téléphone et radio~: 8~bits 
\item 
  le \emph{nombre de voies} ou \emph{channel}. Il peut y en avoir 1 (mono), 2
  (stéréo), 6 (5.1) voire plus. 
\end{itemize}

\begin{enumerate}
\item 
  Quelle est la taille des données à stocker pour enregistrer
  10~minutes de musique sur un CD~? 
\end{enumerate}

% 10 minutes = 600 secondes donc 600*44100 valeurs
% chaque valeur est codée sur 16 bits soit 2 octets
% le son est en stéréo
% donc 600*44100*2*2 = 105 840 000 octets
% on divise par 1024 : 103 359 kO
% on divise encore par 1024 : 100,1 MO


%\newpage

\subsection*{Manipulation de fichiers son avec Python}

Il existe de multiples normes, et de nombreux formats de fichiers pour
stocker des sons numérisés. Certains formats sont compressés, comme
\texttt{.mp3} ou \texttt{.aac}. D'autres ne sont pas compressés
comme \texttt{.wav} ou \texttt{.aif}. \\
Nous travaillons ici avec des fichiers \texttt{.wav}, que nous lirons
et écrirons grâce aux deux fonctions \verb#scipy.io.wavfile.read# et
\verb#scipy.io.wavfile.write#. 

La fréquence d'échantillonnage correspond donc à \verb#rate#.\\
La
résolution est donnée par le \verb#dtype# du tableau de données~: 
\verb#int8# ou \verb#int16#. \\
Le
nombre de voies est donné par le \verb#shape# du tableau de données~:
\verb#(n,)# ou \verb#(n,2)#. 

%\newpage

\subsection*{Les questions}

\begin{enumerate}[resume*]
\item 
  Déterminer, pour les trois fichiers à télécharger sur le site de la
  classe, les caractéristiques du son (nombre de voies, fréquence,
  résolution). Préciser également la durée en secondes.  
\item 
  Définir une fonction \verb#affiche(filename)# qui affiche la
  première voie du signal sur un graphique. 
\item 
  Définir une fonction
  
  \verb#ajoute_silence(data, rate, duree_ms, debut=True)# 
  qui prend en arguments un tableau numpy de son mono, sa fréquence
  d'échantillonnage, la durée en ms du silence à ajouter, et un
  booléen indiquant si le silence doit être ajouté au début ou à la
  fin~; et qui renvoie le tableau numpy de son où un silence a été
  ajouté. 
\item 
  Définir une fonction
  
  \verb#mono_to_stereo(data, rate, duree_ms = 20)#
  qui prend en argument un tableau numpy de son, sa fréquence
  d'échantillonnage~; et renvoie un tableau numpy de son correspondant
  à deux voies identiques, mais la seconde sera décalée de 20 ms.
\item 
  Définir une fonction
  
  \verb#mixage(data, sound, rate, time, voie = 0)#
  qui prend en argument deux tableaux de son (l'un stéréo, l'autre
  mono), 
  la fréquence d'échantillonnage, un temps en secondes et le numéro
  d'une voie~; et renvoie un tableau numpy de son
  correspondant à \verb#data# où a été superposé \verb#sound# à partir
  de l'instant \verb#time#~s sur la voie~\verb#voie#. \\
  On ajoutera simplement les deux ondes sonores, sans se préoccuper de
  l'augmentation du gain et du risque de saturation. 
\item 
  Utiliser les fonctions précédentes pour créer un fichier stéréo,
  avec le bruit de la mer, un cri de mouette à gauche après 4
  secondes, un cri de mouette à droite après 12 secondes, puis, un peu
  plus tard, une sirène de bateau sur les deux voies. 

\item 
  Dégrader l'un des fichiers son pour qu'il soit au standard de la
  téléphonie. 
  % https://en.wikipedia.org/wiki/Decimation_(signal_processing)
\end{enumerate}

\vspace*{1cm}

\centerline{\includegraphics[width=\linewidth]{images/theme_son_2_fig_3}}

 \fi
%
%\section*{Exercice  -- }
%
%\ifprof 
%\begin{corrige}
%\begin{python}
%\end{python}
%\end{corrige}
%\else
%\fi
%
%
%\section*{Exercice  -- }
%
%\ifprof 
%\begin{corrige}
%\begin{python}
%\end{python}
%\end{corrige}
%\else
%\fi
%
%
%\section*{Exercice  -- }
%
%\ifprof 
%\begin{corrige}
%\begin{python}
%\end{python}
%\end{corrige}
%\else
%\fi
%

\ifprof
\else
\end{multicols}
\fi

\ifprof
\else
%\newpage
\footnotesize
\vspace{1cm}
\begin{multicols}{2}
\begin{python}
Help on function read in module scipy.io.wavfile:

read(filename, mmap=False)
    Return the sample rate (in samples/sec) and data 
    from a WAV file    

    Parameters
    ----------
    filename : string or open file handle
        Input wav file.
    mmap : bool, optional
        Whether to read data as memory mapped.
        Only to be used on real files (Default: False)
    Returns
    -------
    rate : int
        Sample rate of wav file
    data : numpy array
        Data read from wav file    
    Notes
    -----    
    * The file can be an open file or a filename.
    * The returned sample rate is a Python integer
    * The data is returned as a numpy array with a
      data-type determined from the file.  
\end{python}

\vfill\null
\columnbreak

\begin{python}
Help on function write in module scipy.io.wavfile:

write(filename, rate, data)
    Write a numpy array as a WAV file
    
    Parameters
    ----------
    filename : string or open file handle
        Output wav file
    rate : int
        The sample rate (in samples/sec).
    data : ndarray
        A 1-D or 2-D numpy array of either integer 
        or float data-type.    
    Notes
    -----
    * The file can be an open file or a filename.
    * Writes a simple uncompressed WAV file.
    * The bits-per-sample will be determined by 
       the data-type.
    * To write multiple-channels, use a 2-D array 
      of shape (Nsamples, Nchannels).  
\end{python}
\end{multicols}
\fi
\ifprof

\else
\fi


\ifprof 
\begin{corrige}
\begin{python}
#! /usr/local/bin/python3.4
# -*- coding:utf-8 -*-

import numpy as np
import matplotlib.pyplot as plt
import scipy.io.wavfile

# (a)
# Pour stocker 10 minutes de musique sur un CD : 
# 2 voies (stéréo), 16 bits, 44100Hz
# donc 10*60*44100*16*2 = 846720000 bits
# soit 105840000 octets
# soit 103359 kO
# soit 100.94 MO

# (b)
for filename in ["seagull.wav", "sea.wav", "horn.wav"]:
    rate, data = scipy.io.wavfile.read(filename)
    print("Pour {} : ".format(filename))
    print("nombre de voies : data.ndim = {} donc le son est mono".format(data.ndim))
    print("fréquence : rate = {} Hz".format(rate))
    print("résolution : data.dtype = {} donc la résolution est 16 bits".format(data.dtype))
    print("durée : {} secondes".format(len(data)/rate))
    print("=====")

# Pour seagull.wav : 
# nombre de voies : data.ndim = 1 donc le son est mono
# fréquence : rate = 22050 Hz
# résolution : data.dtype = int16 donc la résolution est 16 bits
# durée : 12.345714285714285 secondes
# =====
# Pour sea.wav : 
# nombre de voies : data.ndim = 1 donc le son est mono
# fréquence : rate = 22050 Hz
# résolution : data.dtype = int16 donc la résolution est 16 bits
# durée : 37.53356009070295 secondes
# =====
# Pour horn.wav : 
# nombre de voies : data.ndim = 1 donc le son est mono
# fréquence : rate = 22050 Hz
# résolution : data.dtype = int16 donc la résolution est 16 bits
# durée : 1.1014512471655329 secondes
# =====
    
# (c)
def affiche(filename):
    """filename désigne un fichier .wav
    représente graphiquement la première voie du signal sur un graphique"""
    rate, data = scipy.io.wavfile.read(filename)
    # on commence par ne conserver que la première voie si le son est stéréo
    if data.ndim > 1:
        data = data[:,0]
    x = np.arange(len(data))
    y = data
    plt.plot(x,y)
    plt.show()        

# affiche("seagull.wav")    

# On peut aller un peu plus loin, et avoir en abscisse des temps
# et en ordonnées des valeurs entre -1 et 1, en fonction de la résolution

def affiche(filename):
    """filename désigne un fichier .wav
    représente graphiquement la première voie du signal sur un graphique"""
    rate, data = scipy.io.wavfile.read(filename)
    # on commence par ne conserver que la première voie si le son est stéréo
    if data.ndim > 1:
        data = data[:,0]
    duree_du_son = len(data)/rate         # durée du son en secondes
    x = np.arange(0, duree_du_son, 1/rate)# les valeurs de temps,
                                          # espacées de 1/fréquence
    resolution = str(data.dtype)          # on regarde la résolution 
    if resolution[-1:] == "8":            # dans le dtype
        y = data / 2**8
    else:
        y = data / 2**(int(resolution[-2:]))
    plt.figure()
    plt.plot(x,y)
    plt.grid()
    plt.title("Représentation du son de {}".format(filename))
    # plt.show()
    plt.savefig(filename[:-4]+".pdf")

# affiche("seagull.wav")
# $\includegraphics[width=.5\textwidth]{../input_exos_python/theme_son_2_fig_5.pdf}$
# affiche("horn.wav")    
# $\includegraphics[width=.5\textwidth]{../input_exos_python/theme_son_2_fig_6.pdf}$
# affiche("sea.wav")    
    
# (d)
def ajoute_silence (data, rate, duree_ms, debut=True ):
    """ Renvoie un tableau de son mono identique à data
    où un silence de durée duree_ms a été ajout au début (à la fin)"""
    assert data.ndim == 1, "attention, le son proposé n'est pas mono"
    silence = np.zeros(((duree_ms/1000)*rate,), dtype=data.dtype)
    if debut:
        b = np.concatenate((silence, data), axis=0)
    else:
        b = np.concatenate((data, silence), axis=0)
    return b

rate, data = scipy.io.wavfile.read("sea.wav")
silence_sea = ajoute_silence(data, rate, 1000, True)
print("data.dtype = {}".format(silence_sea.dtype))
# data.dtype = int16
print("durée : {} secondes".format(len(silence_sea)/rate))
# durée : 38.53356009070295 secondes

scipy.io.wavfile.write("silence_sea.wav", rate, silence_sea)
# affiche("silence_sea.wav")
# $\includegraphics[width=.5\textwidth]{../input_exos_python/theme_son_2_fig_7.pdf}$
sea_silence = ajoute_silence(data, rate, 1000, False)
scipy.io.wavfile.write("sea_silence.wav", rate, sea_silence)
# affiche("sea_silence.wav")

# (e)
def mono_to_stereo(data, rate, duree_ms = 20):
    """Convertit un son mono en stéréo
    selon la méthode proposée"""
    assert data.ndim == 1, "attention, le son proposé n'est pas mono"
    voie1 = ajoute_silence(data, rate, duree_ms, True)
    voie2 = ajoute_silence(data, rate, duree_ms, False)
    n = len(voie1)
    voie1.shape = (n, 1)
    voie2.shape = (n, 1)
    data_stereo = np.concatenate((voie1, voie2), axis = 1)
    # print("Création d'un tableau de son de shape {}".format(data_stereo.shape))
    return data_stereo

rate, data = scipy.io.wavfile.read("sea.wav")
data_stereo = mono_to_stereo(data, rate)
# Création d'un tableau de son de shape (828056, 2)
scipy.io.wavfile.write("sea_stereo.wav", rate, data_stereo)

# (f)
def mixage(data, sound, rate, time, voie=0):
    """Ajoute à data le son sound à partir de l'instant time en secondes"""
    assert data.ndim == 2, "attention, le son proposé n'est pas stéréo"
    debut = time*rate
    data_mixee = np.copy(data)
    data_mixee[debut:debut+len(sound),voie] += sound
    return data_mixee

# (g)
rate, data = scipy.io.wavfile.read("sea.wav")
_, mouette = scipy.io.wavfile.read("seagull.wav")
_, klaxon = scipy.io.wavfile.read("horn.wav")
data_stereo = mono_to_stereo(data, rate)
data1 = mixage(data_stereo, mouette, rate, 4, 0)
data2 = mixage(data1, mouette, rate, 12, 1)
data3 = mixage(data2, 1.8*klaxon, rate, 20, 0)   # je force un peu le son 
data4 = mixage(data3, 1.8*klaxon, rate, 20, 1)   # de la sirène de bateau

scipy.io.wavfile.write("son_mixe.wav", rate, data4)

# (h)
# D'une part, il faut rééchantillonner le signal.
# Comme les fréquences ne sont pas multiples l'une de l'autre,
# on peut supprimer régulièrement des valeurs, mais la qualité
# sonore est alors très altérée.
# On calcule donc plutot une moyenne pondérée des valeurs
# encadrant l'instant théorique du rééchantillonnage.
# $\includegraphics[width=\textwidth]{../input_exos_python/theme_son_2_fig_4}$
# Peut-on éviter une boucle ici, je ne pense pas. 

def reechantillonnage(filename, newfilename, newrate):
    """Convertit le fichier filename, échantillonné avec une fréquence donnée
    en le fichier newfilename, échantillonné avec une fréquence newrate"""
    rate, data = scipy.io.wavfile.read(filename)
    newdata = np.zeros( (len(data)*newrate//rate,) , dtype=np.int16)

    for i in range(len(newdata)):
        t = i / newrate
        j = t * rate
        k = np.floor(j)

        newdata[i] =  (j-k)*data[k] + (k+1-j)*data[k+1]

    scipy.io.wavfile.write(newfilename, newrate, newdata)
    return None

reechantillonnage("seagull.wav", "seagull_2.wav", 8000)

# D'autre part, il faut réduire la résolution du signal, pour passer en int8
# Mais attention, en lisant la doc de wavfile.write, on voit qu'il faut
# passer en uint8 et pas en int8

def reduction_resolution(filename, newfilename):
    """Réduit la résolution int16 du son du fichier filename
    en int8, dans le fichier newfilename"""
    rate, data = scipy.io.wavfile.read(filename)
    newdata=(128+(data/2**16)*2**8).astype(np.uint8)
    print(newdata.dtype)
    print(newdata.max(),newdata.min())

    scipy.io.wavfile.write(newfilename, rate, newdata)
    return None

reduction_resolution("seagull_2.wav", "seagull_3.wav")

\end{python}
\end{corrige}
\else
\fi

\end{document}
\section*{Exercice  -- }


\ifprof 
\else
\fi


\section*{Exercice}
\setcounter{exo}{0}
\subparagraph{}
\textit{}
\ifprof
\begin{corrige}
\end{corrige}
\else
\fi