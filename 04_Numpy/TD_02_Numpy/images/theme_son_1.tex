% gh

Dans cet exercice, on souhaite juste s'approprier la manipulation des
tableaux \texttt{numpy}.

On définit les tableaux suivants~: 
\begin{python}
import numpy as np

a = np.array([1, 2, 3])
b = np.array([4, 5, 6])  
c = np.array([1, 2, 3, 4, 5, 6])
d = np.array([[1, 2, 3],
              [4, 5, 6]])
e = np.array([[1, 4],
              [2, 5],
              [3, 6]])
\end{python}

\begin{enumerate}
\item 
  Quels sont les \texttt{ndim}, \texttt{shape}, \texttt{dtype}, \texttt{size} de ces
  tableaux~? Utilise-t-on \texttt{len} sur un tableau~? 
\item 
  Construire, par concaténation de \verb#a# et \verb#b#, un tableau \verb#f#
  égal à \verb#c#.
\item 
  Modifier l'attribut \texttt{shape} de \verb#f# pour qu'il soit égal
  à \verb#d#. Vérifier cette égalité. 
\item
  Modifier \verb#f# pour que les éléments
  soient des flottants. Le tableau \verb#f# est-il encore égal à \verb#d#~?
\item 
  Construire, par modification de l'attribut \texttt{shape} et par
  concaténation, à partir de \verb#a# et \verb#b#, un tableau \verb#g#
  égal à \verb#e#.
\item 
  Construire, à partir du tableau \verb#e# et de la fonction
  \texttt{np.zeros}, un tableau \verb#h# représentant 
  $
  \begin{bmatrix}
    1 & 4 & 0 & 0 \\
    2 & 5 & 0 & 0 \\
    3 & 6 & 0 & 0 
  \end{bmatrix}$. 
  Ce tableau aura le même \texttt{dtype} que \verb#e#. 
  
\end{enumerate}
